\documentclass[a4paper,12pt]{article}

% defining margin for the document
\usepackage[margin=1.00in]{geometry}

% these packages are used for images
\usepackage{graphicx}

\title{Problem 1: Gamma function, $\Gamma$ (x)}
\author{Saraswati Saud \\
Student ID: 40115097}
\date{}
\begin{document}
\maketitle
\section{Introduction}
    \subsection{Description}
    The gamma function represented by $\Gamma$(x), is an extension of the factorial function to complex numbers. It is defined for all the complex numbers except the non-positive integers.\\ \\
    Specifically, $\Gamma$(n) = (n-1)! where n $\textgreater$ 0.

    \subsection{Domain}
    Any complex number that is not a negative integer is in the domain of the function.\\
    So, the domain of the Gamma function is (0, +$\infty$).
    
    \subsection{Co-domain}
    The co-domain is the set of all real numbers (-$\infty$, +$\infty$).
    
    \subsection{Characteristics of Gamma Function}
    \begin{enumerate}
        \item The gamma function is uniquely defined for all positive integers and complex numbers with positive real parts.
        \item For real values of argument ‘n’, the value of the gamma function $\Gamma$(n) are real (or infinity). The gamma function is not equal to zero.
        \item $\Gamma$(n+1) = n!, for integer n $\textgreater$ 0.
        \item $\Gamma$(n+1) = n $\Gamma$(n) (function equation).
    \end{enumerate}
    
    \begin{thebibliography}{}
        \bibitem{}
        MathWorld: Gamma Function
        \\\texttt{https://mathworld.wolfram.com/GammaFunction.html}
        \bibitem{}
        Geeksforgeeks
        \\\texttt{https://www.geeksforgeeks.org/gamma-function}
    \end{thebibliography}
\end{document}
