\documentclass{article}
\usepackage[utf8]{inputenc}
\usepackage{graphicx}
\title{soen6011-P3-C}

\title{Problem 3 Pseudocode Format}
\author{Arnab Roy, Saraswati Saud, Shagun Sharma, Pavit Srivatsan }
\date{}
\begin{document}

\maketitle

\section{Guidelines}
\begin{itemize}
\item Always capitalize the initial word (often one of the main 6 constructs).\\
\item Have only one statement per line.\\
\item Indent to show hierarchy, improve readability, and show nested constructs.\\
\item Always end multiline sections using any of the END keywords (ENDIF, ENDWHILE, etc.).\\
\item Keep your statements programming language independent.
\item Use the naming domain of the problem, not that of the implementation. E.g., “Append the last name to the first name” instead of “name = first+ last.”\\
\item Keep it simple, concise, and readable.\\
\end{itemize}
\section{Mindmap}\\
Meetings were held and pseudocode format was finalized based on the above mentioned guidelines. \\
\begin{thebibliography}{9}
\bibitem{pseudocode format}
Pseudocode Format,\\
\url{https://towardsdatascience.com/pseudocode-101-an-introduction-to-writing-good-pseudocode-1331cb855be7}
\end{thebibliography}
\end{document}
