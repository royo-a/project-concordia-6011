\documentclass{article}
\usepackage[utf8]{inputenc}

\title{SOEN6011- P4}
\author{Pavit Srivatsan\\ 40155323}
\date{August 2021}

\begin{document}

\maketitle

\section{Debugging}
Debugging is the process of detecting and removing existing and potential errors in a software code that can cause it to behave unexpectedly or crash.
\subsection{Debugging the program in Eclipse IDE}
\subsection*{A.Setting up breakpoints}
Suitable breakpoints are chosen. We must be careful in choosing
breakpoints to understand the program flow. There are certain cases in which breakpoints can be skipped based on the program logic.
\subsection*{B.Debug Configurations}
A Java program can be debugged simply by right-clicking on the Java editor class file from Package explorer. Select Debug As → Java Application or use the shortcut Alt + Shift + D, J instead. 
\subsection*{C.Debug perspective}
In debug perspective we can skip breakpoints, step into, step over or resume statement executions. We can modify variable values in the variables tab in debug perspective.
\subsection{Advantages}
\begin{itemize}
    \item we understand program flow and logic better
    \item Identify errors and bugs better instead of printing variable values
    \item Modify the program to be more efficient
\end{itemize}
\section{Quality Attributes}
\subsection{Correctness}
The values are more precise. Since the calculator is scientific in nature, correct values are mandatory. Results are accurate as possible. 
\subsection{Efficiency}
The algorithm chosen is the Divide and Conquer Algorithm which is more efficient than the Iterative algorithm. The time complexity of this algorithm (O(log(n))) is better than Iterative (O(n)).
\subsection{Maintainability}
The function is developed from scratch and hence it requires helper functions. These functions are placed appropriately. Necessary comments are added to make it more maintainable.
\subsection{Robustness}
All relevant errors are added so that users are directed to enter proper inputs. It handles various errors and exceptions.
\subsection{Usability}
The program is run in a menu driven model. Menu driven model is easy to use and comprehend. 
\section{CheckStyle}
Checkstyle is a development tool to help programmers write Java code that adheres to a coding standard. It automates the process of checking Java code to spare humans of this boring (but important) task. This makes it ideal for projects that want to enforce a coding standard [1].
\subsection{Features}
Checkstyle can check many aspects of your source code. It can find class design problems, method design problems. It can check code layout and formatting issues[1].
\subsection{Advantages}
\begin{itemize}
    \item portable between IDEs. If you decide to use IntelliJ later, or you have a team using a variety of IDEs, you still have a way to enforce consistency [3].
    \item better external tooling. It's much easier to integrate checkstyle with your external tools since it was designed as a standalone framework. You can plug into your SCM as a pre-commit hook, or into your build tool, quite easily. Using Eclipse style convention you would need to write or locate a plugin to do the same thing [3].
    \item ability to create your own rules. Eclipse defines a large set of styles, but checkstyle has more, and you can add your own custom rules [3].
\end{itemize}
\subsection{Limitations}
There are basically only a few limits for Checkstyle:
\begin{itemize}
\item Java tokens (identifiers, keywords) should be written with ASCII characters ONLY, no Unicode escape support in keywords and no Unicode support in identifiers.[1]
\item To get valid violations, code have to be compilable, in other case you can get not easy to understand parse errors.[1]
\item You cannot determine the type of an expression. Example: "getValue() + getValue2()".[1]
\item You cannot determine the full inheritance hierarchy of type.[1]
\item You cannot see the content of other files. You have content of one file only during all Checks execution. All files are processed one by one. [1]
\end{itemize}
\begin{thebibliography}{9}
\bibitem{checkstyle}
Checkstyle,\\
\url{https://checkstyle.sourceforge.io/index.html}

\bibitem{debugging}
Debugging,\\
\url{https://www.eclipse.org/community/eclipse_newsletter/2017/june/article1.php}

\bibitem{stackoverflow}
Stackoverflow,\\
\url{https://stackoverflow.com/questions/13644624/advantage-of-using-checkstyle-rather-than-using-eclipse-built-in-code-formatter}
\end{thebibliography}
\end{document}

